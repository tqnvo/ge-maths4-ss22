\documentclass[12pt]{article}
\usepackage{anysize}
\usepackage[ngerman,english]{babel}
\marginsize{3.5cm}{2.5cm}{1cm}{2cm}

\usepackage[utf8]{inputenc}
\usepackage[english]{babel}
\usepackage{amsmath}
\usepackage{amsthm}
\usepackage{amsfonts}
\usepackage{scalerel,amssymb}  % for mathbb
%---------------------------------------------------------
\usepackage{authblk}
\usepackage{blindtext}
\usepackage{graphicx}
\usepackage[caption=true]{subfig}
\usepackage[
  colorlinks = true,
  citecolor = red,
  % linkcolor = darkblue, % internal references
  % urlcolor = darkblue,
]{hyperref}
\usepackage{cleveref}
\crefname{figure}{Figure}{Figures}
\crefname{equation}{Eq.}{Eqs.}
%---------------------------------------------------------
% \newcommand{\begin{equation}}{\eqbeg} 
% \newcommand{\end{equation}}{\eqend}
\usepackage{bm}
\usepackage{dfcmd}
% \newcommand{\Bvarepsilon}{\bm\varepsilon} 
% \newcommand{\Bx}{\bm x} 
% \newcommand{\Bu}{\bm u} 
% \newcommand{\Bv}{\bm v} 
% \newcommand{\Bg}{\bm g} 
% \newcommand{\Bb}{\bm b} 
\newcommand{\divv}{\text{divv}} 
%---------------------------------------------------------
\usepackage{lineno}
% \linenumbers     

\usepackage{float}
\usepackage{psfrag}

\usepackage{fontawesome}
\usepackage{relsize}

%---------------------------------------------------------
%For \toprule \midrule \bottomrule in table environment
\usepackage{booktabs}

%---------------------------------------------------------
%For cancelto
\usepackage[makeroom]{cancel}

%For \mathghost \mathleftghost \mathrightghost
\usepackage{halloweenmath}
%Page numbering in style 1/3...
\usepackage{lastpage}  
\usepackage{hyperref}
\makeatletter
\renewcommand{\@oddfoot}{\hfil 
% Aachen, November $04^{th}$, 2021 \hspace{300pt} 
Mathe IV $\cdot$ Gue10 $\cdot$ SS22 
\hspace{280pt} 
\thepage/\pageref{LastPage}\hfil}
\makeatother
%------------------------------------------------------------------------------
\begin{document}
\begin{center}
	\section*{Global Exercise - Gue10}
\end{center}
\begin{center}
	Tuan Vo
\end{center}
\begin{center}
	$23^{\text{rd}}$ June, 2022
\end{center}
Content to be covered:
\begin{itemize}
	\item[\checkmark] Programming exercise 02: Explanation.
	      % \item[\checkmark] Analysis:
	      %       \begin{itemize}
	      % 	      \item[*] Distributional derivative.
	      %   \end{itemize}
	\item[\checkmark] [Q\&A] Insight review: Example 3 $\cdot$ Gue08
	\item[\checkmark] Euler scheme
	      $\begin{cases}
			      \text{Explicit} \\
			      \text{Implicit}
		      \end{cases}
		      :
		      \sum
		      \longrightarrow
		      \text{Crank-Nicolson scheme}
	      $
	\item[\checkmark] Iterative solution method
	\item[\checkmark] [Review HW10] Consistency error
\end{itemize}

%------------------------------------------------------------------------------
\section{Programming exercise 02: Explanation}
\begin{example}
	Examine the following Convection-diffustion problem
	\begin{align*}
		-\varepsilon\Delta u(x,y)
		+ \cos\beta\;\frac{\partial u}{\partial x}(x,y)
		+ \sin\beta\;\frac{\partial u}{\partial y}
		       & = f(x,y), \ (x,y) \in \Omega = (0,1)^2,          \\
		u(x,y) & = g(x,y), \ (x,y) \ \text{on} \ \partial \Omega,
	\end{align*}
\end{example}
Main points of discretization:
\begin{align*}
	-\varepsilon\,  \partial_{xx} u(x,y)
	-\varepsilon\,  \partial_{yy} u(x,y)
	+\cos\beta\, \partial_{x} u(x,y)
	+\sin\beta\, \partial_{y} u(x,y) = 1
\end{align*}
Since $\beta = 5\pi/6$, it goes with $\cos\beta < 0$ and $\sin\beta > 0$.
Therefore, we need 
Upwind for $\partial_{x} u(x,y)$ and 
Downwind for $\partial_{y} u(x,y)$, 
as follows
\begin{enumerate}
	\item Central difference method applied for $\partial_{xx} u(x,y)$,
	\item Central difference method applied for $\partial_{yy} u(x,y)$,
	\item Upwind method applied for $\partial_{x} u(x,y)$,
	\item Downwind method applied for $\partial_{y} u(x,y)$.
\end{enumerate}
\begin{align*}
	-\varepsilon\,\frac{u_{i+1,j}-2u_{i,j}+u_{i-1,j}}{h^2}
	 & -\varepsilon\,\frac{u_{i,j+1}-2u_{i,j}+u_{i,j-1}}{h^2} \\
	 & +\cos\beta\,\frac{u_{i+1,j}-u_{i,j}}{h}
	+\sin\beta\,\frac{u_{i,j}-u_{i,j-1}}{h}
	=1
\end{align*}
\begin{align*}
	\dots
	[\text{further steps is followed by Problem 2.41 in the lecture note.}]
\end{align*}
%------------------------------------------------------------------------------
\newpage
\section{[Q\&A] Insight review: Example 3 $\cdot$ Gue08}
\begin{example}
	\label{ex:example2consistencyorder}
	Examine the consistency order of the following problem
	\begin{align*}
		f'(x) \approx f(x) + f(x+h) + f(x+2h).
	\end{align*}
	(the consistency order is aimed as high as possible)
\end{example}
Approach: 
\begin{enumerate}
	\item Let's expand $f(x+h)$ and $f(x+2h)$
	      by using Taylor expansion till $\mathcal{O}(h^3)$
	      \begin{align}
		      \label{eq:firstfhreeTaylor}
		      \begin{cases}
			      \begin{aligned}
				      f(x)    & = f(x)                \\
				      f(x+ h) & = f(x) 
				      + \frac{h^1}{1!} f'(x)  
				      + \frac{h^2}{2!} f''(x)  
				      + \mathcal{O}\left( h^3 \right) \\
				      f(x+2h) & = f(x) 
				      + \frac{(2h)^1}{1!} f'(x) 
				      + \frac{(2h)^2}{2!} f''(x) 
				      + \mathcal{O}\left( (2h)^3 \right)
			      \end{aligned}
		      \end{cases}
	      \end{align}
	      which leads to 
	      \begin{align}
		      \alpha f(x)    & = \alpha f(x)  \label{eq:alphafx}                    \\
		      \beta f(x+ h)  & = \beta f(x) 
		      + \beta h f'(x)  
		      + \frac{\beta h^2}{2} f''(x)
		      + \beta \mathcal{O}\left( h^3 \right)              \label{eq:betafxh} \\
		      \gamma f(x+2h) & = \gamma f(x) 
		      + 2\gamma h f'(x)  
		      + 2\gamma h^2 f''(x)
		      + \gamma \mathcal{O}\left( (2h)^3 \right) \label{eq:gammafx2h}
	      \end{align}
	      Summation of \eqref{eq:alphafx}, \eqref{eq:betafxh} and \eqref{eq:gammafx2h} leads to
	      \begin{align}
		      \label{eq:summationall}
		      \alpha f(x) + \beta f(x+h) + \gamma f(x+2h) 
		       & = (\alpha + \beta + \gamma)f(x) \notag                                \\
		       & \quad + (\beta h + 2 \gamma h)f'(x) \notag                            \\
		       & \quad + \left( \frac{\beta h^2}{2} + 2\gamma h^2 \right)f''(x) \notag \\
		       & \quad + \beta \mathcal{O}\left( h^3 \right)                    \notag \\
		       & \quad + \gamma \mathcal{O}\left( (2h)^3 \right)
	      \end{align}
	      % Comparison of \eqref{eq:summationall} with \eqref{eq:givenproblemfxprime}
	      Since we would like to approximate $f'(x)$ by the following expression
	      \begin{align}
		      \label{eq:wewouldliketoapprox}
		      \alpha f(x) + \beta f(x+h) + \gamma f(x+2h) \approx f'(x)
	      \end{align}
	      Comparison \eqref{eq:wewouldliketoapprox} with \eqref{eq:summationall} leads to the following 3 conditions
	      \begin{align}
		      \label{eq:alphabetagamma}
		      \begin{cases}
			      \alpha + \beta + \gamma \stackrel{!}{=} 0 \\
			      \beta h + 2 \gamma h \stackrel{!}{=}  1   \\
			      \displaystyle \frac{\beta h^2}{2} + 2\gamma h^2 \stackrel{!}{=} 0
		      \end{cases}
		       & \Leftrightarrow
		      \begin{pmatrix}
			      1 & 1   & 1 \\
			      0 & 1   & 2 \\
			      0 & 1/2 & 2
		      \end{pmatrix}
		      \begin{pmatrix}
			      \alpha \\ \beta \\ \gamma
		      \end{pmatrix}
		      =
		      \begin{pmatrix}
			      0 \\ 1/h \\ 0
		      \end{pmatrix} \\
		       & \Leftrightarrow
		      \begin{pmatrix}
			      \alpha \\ \beta \\ \gamma
		      \end{pmatrix}
		      =
		      \begin{pmatrix}
			      -3/(2h) \\ 2/h \\ -1/(2h)
		      \end{pmatrix}
	      \end{align}
	      which leads the expression \eqref{eq:summationall} to the following equality
	      \begin{align}
		      \label{eq:finalsum}
		      \alpha (x) + \beta f(x+h) + \gamma f(x+2h) 
		       & =
		      f'(x)
		      + \beta \mathcal{O}\left( h^3 \right) + \gamma \mathcal{O}\left( (2h)^3 \right) \notag \\
		       & = f'(x)
		      + \frac{2}{h} \mathcal{O}\left( h^3 \right)
		      + \frac{-1}{2h} \mathcal{O}\left( (2h)^3 \right)                                \notag \\
		       & = f'(x) + \mathcal{O}\left( h^2 \right).
	      \end{align}
	      Therefore, $f'(x)$ can be approximated by $\alpha f(x) + \beta f(x+h) + \gamma f(x+2h)$,
	      where the values of $\left\{ \alpha, \beta, \gamma \right\}$ taken from \eqref{eq:alphabetagamma}, 
	      with a consistency order of at least order 2, as shown in \eqref{eq:finalsum}.
	      
	\item Let's now expand $f(x+h)$ and $f(x+2h)$
	      by using Taylor expansion till $\mathcal{O}(h^4)$.
	      The approach is the same as the above approach  
	      for $\mathcal{O}(h^3)$.
	      However, this will lead to contradictory conditions,  
	      i.e. 3 unknowns $\left\{ \alpha, \beta, \gamma \right\}$  for 4 equations. 
	      
	\item Therefore, the highest consistency order we may obtain is of order 2.
\end{enumerate}

%------------------------------------------------------------------------------
\newpage
\begin{example}
	Examine the consistency order of the following problem
	\begin{align*}
		f''(x) \approx f(x) + f(x+h) + f(x+2h).
	\end{align*}
	(the consistency order is aimed as high as possible)
\end{example}
Approach: The procedure goes as same as Example \ref{ex:example2consistencyorder}
\begin{enumerate}
	\item Let's expand $f(x+h)$ and $f(x+2h)$
	      by using Taylor expansion till $\mathcal{O}(h^3)$
	      \begin{align}
		      \label{eq:firstfhreeTaylor}
		      \begin{cases}
			      \begin{aligned}
				      f(x)    & = f(x)                \\
				      f(x+ h) & = f(x) 
				      + \frac{h^1}{1!} f'(x)  
				      + \frac{h^2}{2!} f''(x)  
				      + \mathcal{O}\left( h^3 \right) \\
				      f(x+2h) & = f(x) 
				      + \frac{(2h)^1}{1!} f'(x) 
				      + \frac{(2h)^2}{2!} f''(x) 
				      + \mathcal{O}\left( (2h)^3 \right)
			      \end{aligned}
		      \end{cases}
	      \end{align}
	      which leads to 
	      \begin{align}
		      \alpha f(x)    & = \alpha f(x)  \label{eq:alphafx}                    \\
		      \beta f(x+ h)  & = \beta f(x) 
		      + \beta h f'(x)  
		      + \frac{\beta h^2}{2} f''(x)
		      + \beta \mathcal{O}\left( h^3 \right)              \label{eq:betafxh} \\
		      \gamma f(x+2h) & = \gamma f(x) 
		      + 2\gamma h f'(x)  
		      + 2\gamma h^2 f''(x)
		      + \gamma \mathcal{O}\left( (2h)^3 \right) \label{eq:gammafx2h}
	      \end{align}
	      Summation of \eqref{eq:alphafx}, \eqref{eq:betafxh} and \eqref{eq:gammafx2h} leads to
	      \begin{align}
		      \label{eq:summationall}
		      \alpha f(x) + \beta f(x+h) + \gamma f(x+2h) 
		       & = (\alpha + \beta + \gamma)f(x) \notag                                \\
		       & \quad + (\beta h + 2 \gamma h)f'(x) \notag                            \\
		       & \quad + \left( \frac{\beta h^2}{2} + 2\gamma h^2 \right)f''(x) \notag \\
		       & \quad + \beta \mathcal{O}\left( h^3 \right)                    \notag \\
		       & \quad + \gamma \mathcal{O}\left( (2h)^3 \right)
	      \end{align}
	      % Comparison of \eqref{eq:summationall} with \eqref{eq:givenproblemfxprime}
	      Since we would like to approximate $f''(x)$ by the following expression
	      \begin{align}
		      \label{eq:wewouldliketoapprox}
		      \alpha f(x) + \beta f(x+h) + \gamma f(x+2h) \approx f''(x)
	      \end{align}
	      Comparison \eqref{eq:wewouldliketoapprox} with \eqref{eq:summationall} leads to the following 3 conditions
	      \begin{align}
		      \label{eq:alphabetagamma}
		      \begin{cases}
			      \alpha + \beta + \gamma \stackrel{!}{=} 0 \\
			      \beta h + 2 \gamma h \stackrel{!}{=}  0   \\
			      \displaystyle \frac{\beta h^2}{2} + 2\gamma h^2 \stackrel{!}{=} 1
		      \end{cases}
		       & \Leftrightarrow
		      \begin{pmatrix}
			      1 & 1   & 1 \\
			      0 & 1   & 2 \\
			      0 & 1/2 & 2
		      \end{pmatrix}
		      \begin{pmatrix}
			      \alpha \\ \beta \\ \gamma
		      \end{pmatrix}
		      =
		      \begin{pmatrix}
			      0 \\ 0 \\ 1/h^2
		      \end{pmatrix} \\
		       & \Leftrightarrow
		      \begin{pmatrix}
			      \alpha \\ \beta \\ \gamma
		      \end{pmatrix}
		      =
		      \begin{pmatrix}
			      ?/h^2 \\ ??/h^2 \\ ???/h^2
		      \end{pmatrix}
	      \end{align}
	      which leads the expression \eqref{eq:summationall} to the following equality
	      \begin{align}
		      \label{eq:finalsum}
		      \alpha (x) + \beta f(x+h) + \gamma f(x+2h) 
		       & =
		      f''(x)
		      + \beta \mathcal{O}\left( h^3 \right)
		      + \gamma \mathcal{O}\left( (2h)^3 \right) \notag                                         \\
		       & = f''(x)
		      + \frac{??}{h^2} \mathcal{O}\left( h^3 \right)
		      + \frac{???}{h^2} \mathcal{O}\left( (2h)^3 \right)                                \notag \\
		       & = f''(x) + \mathcal{O}\left( h \right).
	      \end{align}
	      Therefore, $f''(x)$ can be approximated by $\alpha f(x) + \beta f(x+h) + \gamma f(x+2h)$,
	      where the values of $\left\{ \alpha, \beta, \gamma \right\}$ taken from \eqref{eq:alphabetagamma}, 
	      with a consistency order of at least order 1, as shown in \eqref{eq:finalsum}.
	      
	\item Let's now expand $f(x+h)$ and $f(x+2h)$
	      by using Taylor expansion till $\mathcal{O}(h^4)$.
	      The approach is the same as the above approach  
	      for $\mathcal{O}(h^3)$.
	      However, this will lead to contradictory conditions,  
	      i.e. 3 unknowns $\left\{ \alpha, \beta, \gamma \right\}$  for 4 equations. 
	      
	\item Therefore, the highest consistency order we may obtain is of order 1.
\end{enumerate}

%------------------------------------------------------------------------------
\newpage
\section{Euler scheme $\cdot$ Crank-Nicolson scheme}
\begin{example}
	Examine the following problem for $\alpha>0$
	\begin{align}
		\label{eq:diffusion}
		\partial_t u(x,t) = \alpha \;\partial_{xx} u(x,t)+q(x,t), \quad t>0, \; x\in[0,1],
	\end{align}
	with IC $u(x,0) = u_0(x),$ and BC $u(0,t) = u(1,t) = 0.$
\end{example}
Approach: Vertical line method, i.e. we first discretize space $x$, 
we then obtain a system of ODEs w.r.t. time $t$.
\begin{enumerate}
	\item Second order discretization in space
	      \begin{align*}
		      \partial_{xx} u(x_j,t)
		      \approx \frac{u(x_{j+1},t) - 2u(x_j,t) + u(x_{j-1},t)}{h^2}
	      \end{align*}
	      Assume $y_j(t) \approx u(x_j,t)$ yields
	      \begin{align*}
		      y'_j(t) =
		      \frac{\alpha}{h^2}
		      (y_{j+1}(t) - 2y_j(t) + y_{j-1}(t)) + q(x_{j},t).
	      \end{align*}
	      General form $y' = Ay + b(t)$ with
	      \begin{align*}
		      A & = -\frac{\alpha}{h_x^2}
		      \begin{pmatrix}
			      2  & -1     &        &        & 0  \\
			      -1 & 2      & -1     &        &    \\
			         & \ddots & \ddots & \ddots &    \\
			         &        & -1     & 2      & -1 \\
			      0  &        &        & -1     & 2
		      \end{pmatrix}  \\
		      \text{and}                  \\
		      b & =
		      \begin{pmatrix}
			      q(x_1,t) \\
			      \vdots   \\
			      q(x_{n-1},t)
		      \end{pmatrix}.
	      \end{align*}
	      where the IC reads
	      \begin{align*}
		      y_i(0)= u_0(x_i),\quad i=1,\ldots,n-1
	      \end{align*}
	      
	\item
	      \begin{enumerate}
		      \item Explicit Euler
		            \begin{align*}
			            \therefore\quad
			            \boxed{
				            y^{i+1} = y^i + h_t \left( Ay^{i}+b^i \right).
			            }
		            \end{align*}
		            
		      \item Implicit Euler
		            \begin{align*}
			            \therefore\quad
			            \boxed{
				            y^{i+1} = y^i + h_t \left( Ay^{i+1}+b^{i+1} \right).
			            }
		            \end{align*}
		            
		      \item Averaging explicit Euler and implicit Euler leads to Crank-Nicolson
		            \begin{align*}
			            \therefore\quad\boxed{
				            \begin{aligned}
					            y^{i+1}
					             & = y^i + \frac{h_t}{2} \left( Ay^i+Ay^{i+1}+b^i+b^{i+1} \right)                  \\
					             & = y^i + \frac{h_t}{2} A (y^i+y^{i+1})+\frac{h_t}{2} \left( b^i+b^{i+1} \right).
				            \end{aligned}
			            }
		            \end{align*}
	      \end{enumerate}
\end{enumerate}

%------------------------------------------------------------------------------
\newpage
\section{Iterative solution method}
\begin{example}
	Examine the following problem: The iteration scheme
	\begin{align*}
		\begin{pmatrix}
			1         & 0 \\
			-\delta a & 1
		\end{pmatrix}
		x^{k+1} =
		\begin{pmatrix}
			1-\delta & \delta a \\
			0        & 1-\delta
		\end{pmatrix}
		x^{k} + \delta b, \quad \delta \in \mathbb{R}.
	\end{align*}
	is used to solve the following problem 
	\begin{align*}
		\begin{pmatrix}
			1  & -a \\
			-a & 1
		\end{pmatrix}
		x = b, \quad x,b\ \in \mathbb{R}^2,
	\end{align*}
\end{example}

\begin{enumerate}
	\item Determine the matrix $C$ from the iteration matrix $I-CA$
	\item Which value of $a \in \mathbb{R}$ can help this method to be converged
	      in case of $\delta=1$?
\end{enumerate}
Approach: Followed by Bemerkung 3.3. from lecture note.
\begin{enumerate}
	\item The iteration scheme takes the form
	      \begin{align*}
		      \boxed{
			      x^{k+1} = (I-CA) x^{k} + Cb.
		      }
	      \end{align*}
	      which is derived directly from the following iteration setting function 
	      \begin{align}
		      \Phi :
		      \begin{cases}
			      \mathbb{R}^n \rightarrow \mathbb{R}^n, \\
			      x \mapsto \Phi(x):= x + C(b-Ax).
		      \end{cases}
	      \end{align}
	      Moreover, when $x^{*}$ is the solution of the system $Ax=b$,
	      we should obtain the following expression
	      \begin{align*}
		      \Phi(x^{*})
		       & = x^{*} + C\underbrace{(b-Ax^{*})}_{=0} \\
		       & = x^{*}.
	      \end{align*}
	      Back to the main problem, we now examine the following derivation
	      \begin{align}
		      \label{eq:iterationphix}
		      x^{k+1} & =
		      \begin{pmatrix}
			      1         & 0 \\
			      -\delta a & 1
		      \end{pmatrix}^{-1}
		      \begin{pmatrix}
			      1-\delta & \delta a \\
			      0        & 1-\delta
		      \end{pmatrix}
		      x^{k}
		      + \delta \begin{pmatrix}
			      1         & 0 \\
			      -\delta a & 1
		      \end{pmatrix}^{-1}b \\
		              & =
		      \begin{pmatrix}
			      1        & 0 \\
			      \delta a & 1
		      \end{pmatrix}
		      \begin{pmatrix}
			      1-\delta & \delta a \\
			      0        & 1-\delta
		      \end{pmatrix}
		      x^{k}
		      + \delta \begin{pmatrix}
			      1        & 0 \\
			      \delta a & 1
		      \end{pmatrix}b      \\
		              & =
		      \begin{pmatrix}
			      1 -\delta              & \delta a              \\
			      \delta a -  \delta^2 a & \delta^2 a^2+1-\delta
		      \end{pmatrix}
		      x^{k}
		      +  \begin{pmatrix}
			      \delta     & 0      \\
			      \delta^2 a & \delta
		      \end{pmatrix}b
	      \end{align}
	      Moreover, we also expect
	      \begin{align}
		      \label{eq:iterationphi}
		      x^{k+1} = (I-CA) x^{k} + Cb.
	      \end{align}
	      Comparison of \eqref{eq:iterationphi} with \eqref{eq:iterationphix} yields
	      the matrix $C$ as follows
	      \begin{align}
		      \therefore\quad
		      \boxed{
			      C =
			      \begin{pmatrix}
				      \delta     & 0      \\
				      \delta^2 a & \delta
			      \end{pmatrix}
		      }
	      \end{align}
	      
	\item For the case $\delta=1$ we obtain the following iteration matrix $(I-CA)$
	      \begin{align}
		      I-CA & =
		      \begin{pmatrix}
			      0 & a   \\
			      0 & a^2
		      \end{pmatrix},
	      \end{align}
	      Observation: This matrix is of upper triangle. 
	      Hence, its eigenvalues are those on the main diagonal, 
	      i.e. we can recognize immediately the eigenvalues 
	      \begin{align}
		      \begin{cases}
			      \lambda_{1} = 0, \\
			      \lambda_{2} = a^2.
		      \end{cases}
	      \end{align}
	      The iteration method is expected to be converged when
	      the spectral radius of the iteration matrix $(I-CA)$ smaller than 1,
	      i.e. we should have the following condition
	      \begin{align*}
		      \rho(I-CA) < 1 
		      \Leftrightarrow a^2 <1 
	      \end{align*}
	      which leads to
	      \begin{align}
		      \therefore\quad
		      \boxed{
			      -1 < a < 1
		      }
	      \end{align}
\end{enumerate}

%------------------------------------------------------------------------------
\newpage
\section{[Review HW10] Consistency error}
\begin{example}
	Examine the consistency error of the following problem
	\begin{align*}
		u''(x) - u'(x) + u(x) = 2x -1 -x^2
	\end{align*}
	with the exact solution is known, i.e. $u(x) = 1-x^2$.
\end{example}
Approach:
The consistency error reads
\begin{align*}
	\big|\big| -\Delta_{h}u\big|_{\bar{\Omega}_{h}} - f\big|_{\Omega_{h}} \big|\big|	_{\infty}
	=
	\big|\big| A_{h}u\big|_{\Omega_{h}} - f\big|_{\Omega_{h}} \big|\big|_{\infty}
\end{align*}
Observation: The consistency error can be foreseen (and indeed) with the value $0$,
i.e. the numerical solution resembles the exact/analytical solution,
since we have been using the following numerical scheme $+$ 
the given information about the exact solution:
\begin{enumerate}
	\item Second order discretization scheme is used to approximate $u''(x)$.
	\item Second order discretization scheme is used to approximate $u'(x)$.
	\item The exact solution, which is given, is of quadratics.
\end{enumerate}


% Since the term $-h^2/3$ from \eqref{eq:finalsum} is nonzero,
% we only need to expand those two Taylor expansions from \eqref{eq:firstfhreeTaylor}
% till $\mathcal{O}(h^3)$. Therefore, the consistency order, 
% due to the factor $h$ coming along with $f'(x)$, i.e. $\mathcal{O}(h^3)/h$, 
% is guaranteed till at least order 2.

% \begin{align}
% 	\label{eq:firstfhreeTaylororder4}
% 	\begin{cases}
% 		\begin{aligned}
% 			f(x)    & = f(x)   \\
% 			f(x+ h) & = f(x) 
% 			+ \frac{h^1}{1!} f'(x)  
% 			+ \frac{h^2}{2!} f''(x)  
% 			+ \frac{h^3}{3!} f'''(x) 
% 			+ \mathcal{O}(h^4) \\
% 			f(x+2h) & = f(x) 
% 			+ \frac{(2h)^1}{1!} f'(x) 
% 			+ \frac{(2h)^2}{2!} f''(x) 
% 			+ \frac{(2h)^3}{3!} f'''(x) 
% 			+ \mathcal{O}(h^4)
% 		\end{aligned}
% 	\end{cases}
% \end{align}

% \newpage
% \inputfig{floats/fixedUpwind_variatedEpsilon}{fixedUpwind_variatedEpsilon}

% \newpage
% \inputfig{floats/fixedCentral_variatedEpsilon}{fixedCentral_variatedEpsilon}

% \newpage
% \inputfig{floats/fixedUpwind_variatedBeta}{fixedUpwind_variatedBeta}

% %------------------------------------------------------------------------------
% \newpage
% \section{Analysis: 2D Eigenfuctions of Laplace operator}
% \begin{recall}
% 	\hphantom{a}\\
% 	\hphantom{a}\\
% 	\hphantom{a}\\
% 	\hphantom{a}\\
% 	\hphantom{a}\\
% 	\hphantom{a}\\
% 	\hphantom{a}\\
% 	\hphantom{a}\\
% 	\hphantom{a}\\
% 	\hphantom{a}\\
% 	\hphantom{a}\\
% 	\hphantom{a}\\
% \end{recall}
% \begin{example}
% 	\label{ex:example2DEigenfcnLaplace}
% 	Examine the following problem where the squared unit domain defined as
% 	$\Omega = [0,1]^2$ and the function
% 	$f(x,y)$ given as follows
% 	\begin{align*}
% 		f(x,y) = xy (1-x)(1-y).
% 	\end{align*}
% 	Expand the given function $f(x,y)$ to eigenfunctions derived from  
% 	\begin{align*}
% 		-\Delta \phi  
% 		     & = \lambda \phi, \ \text{in} \ \Omega, \\
% 		\phi & = 0, \ \text{on} \ \partial \Omega.
% 	\end{align*}
% \end{example}
% Approach: The Eigenfunctions take the form
% \begin{align*}
% 	\boxed{
% 		\phi_{j,k} (x,y) = C_1 \sin (j \pi x) \sin(k \pi y).
% 	}
% \end{align*}
% Normalization of eigenfunctions goes as follows
% \begin{align*}
% 	\boxed{
% 		\begin{aligned}
% 			||\phi_{j,k} (x,y)||^2_{2} 
% 			\stackrel{!}{=} 1 
% 			 & \Leftrightarrow
% 			\int_\Omega \phi_{j,k}^2(x,y) \ d\Omega 
% 			\stackrel{!}{=} 1   \\
% 			 & \Leftrightarrow 
% 			\int_0^1 \int_0^1 C_1^2 \sin^2 (j \pi x) \sin^2(k \pi y) \ dx dy
% 			\stackrel{!}{=} 1
% 			\Rightarrow C_1=2.
% 		\end{aligned}
% 	}
% \end{align*}
% \newpage
% Derivation of eigenfunctions $\phi_{j,k} (x,y)$

% \newpage
% Derivation of eigenfunctions $\phi_{j,k} (x,y)$ (cont.)

% \newpage
% Expansion
% \begin{align*}
% 	f(x,y) = xy(1-x)(1-y)
% 	\stackrel{!}{=}
% 	\sum_{j=1}^\infty  \sum_{k=1}^\infty \alpha_{j,k} \phi_{j,k} (x,y)
% \end{align*}
% where the coefficient $\alpha_{j,k}$ is computed as follows
% \begin{align*}
% 	\alpha_{j,k} & = \int_\Omega f(x,y) \phi_{j,k} \ d\Omega                                   \\
% 	             & =  \int_0^1 \int_0^1 xy(1-x)(1-y) \ 2 \sin (j \pi x) \sin(k \pi y) \  dx dy \\
% 	             & = 2 \int_0^1  x(1-x) \sin(j \pi x) \ dx \int_0^1  y(1-y) \sin(k \pi y) \ dy
% \end{align*}
% which is computed firstly w.r.t. $dx$ step-by-step as follows
% \begin{align*}
% 	\int_0^1  x(1-x) \sin(j \pi x) \ dx
% 	= \frac{2}{j^3 \pi^3} (1-\cos(j \pi)) - \frac{1}{j^2 \pi^2} \sin(j \pi)
% \end{align*}
% The term goes with sinus function gets vanished, while the term goes with cosinus function takes the form
% $\cos(j \pi) = (-1)^j$. Similarly, it goes the same for $dy$. Finally, we obtain
% \begin{align*}
% 	\alpha_{j,k} & = 2 \frac{2}{j^3 \pi^3} \left( 1-(-1)^j  \right)  \frac{2}{k^3 \pi^3} \left( 1-(-1)^k  \right) \\
% 	             & =
% 	\left\{
% 	\begin{matrix}
% 		\dfrac{32}{j^3 k^3 \pi^6}, & \text{if } j, k \text{ odd}, \\
% 		0,                         & \text{otherwise},
% 	\end{matrix}\right.
% \end{align*}
% which leads to the final form of expansion of $f(x,y)$ as follows
% \begin{align*}
% 	\therefore\quad\boxed{
% 		f(x,y) = \sum_{j,k \ \text{odd}}  \frac{32}{j^3 k^3 \pi^6} 2 \sin(j \pi x ) \sin(k \pi y ).
% 	}
% \end{align*}

% %------------------------------------------------------------------------------
% \newpage
% \section{Analysis: Distributional derivative}
% \begin{example}
% 	Examine the following problem 
% 	\begin{align*}
% 		f:
% 		\begin{cases}
% 			\mathbb{R} \to \mathbb{R}, \\
% 			x \mapsto f(x) = |x+1|.
% 		\end{cases}
% 	\end{align*}
% \end{example}

% %------------------------------------------------------------------------------
% \newpage
% \section{Numerics: Ghost points in FDM}
% \begin{example}
% 	Examine the following FDM problem:
% 	\begin{align*}
% 		u''(x) - u'(x) + u(x) = -x^2 + 2x - 1,
% 		\quad
% 		\text{for }\  x\in \Omega =(0,1),
% 	\end{align*}
% 	\inputfig{floats/discretizeGhost}{discretizeGhost}
% 	where the boundary conditions are defined as follows
% 	\begin{align*}
% 		u(0) = 0, \quad u(1) = 0.
% 	\end{align*}
% 	Determine the resulting system of linear equations 
% 	$A_h u_h = b_h$ by using second order FDM scheme for both $u''(x)$ and $u'(x)$. 
% \end{example}
% Approach: This problem exists no ghost point $\mathghost$\\

% The second order FDM scheme for both $u''(x)$ and $u'(x)$ take the following forms
% \begin{align*}
% 	\boxed{
% 		\begin{aligned}
% 			u''(x) & \approx  \frac{u(x+h)-2u(x)+u(x-h)}{h^2}, \\
% 			u'(x)  & \approx   \frac{u(x+h)-u(x-h)}{2h},
% 		\end{aligned}
% 	}
% \end{align*}

% %------------------------------------------------------------------------------
% \newpage
% \begin{example}
% 	Examine the following FDM problem:
% 	\begin{align*}
% 		u''(x) - u'(x) + u(x) = -x^2 + 2x - 1,
% 		\quad
% 		\text{for }\  x\in \Omega =(0,1),
% 	\end{align*}
% 	\inputfig{floats/discretizeGhostLeft}{discretizeGhostLeft}
% 	where the boundary conditions are defined as follows
% 	\begin{align*}
% 		u'(0) = 0, \quad u(1) = 0.
% 	\end{align*}
% 	Determine the resulting system of linear equations 
% 	$A_h u_h = b_h$ by using second order FDM scheme for both $u''(x)$ and $u'(x)$. 
% \end{example}

% Approach: This problem exists a ghost point $\mathleftghost$
% on the LHS boundary $\partial\Omega_{L}$\\

% The second order FDM scheme for both $u''(x)$ and $u'(x)$ take the following forms
% \begin{align*}
% 	\boxed{
% 		\begin{aligned}
% 			u''(x) & \approx  \frac{u(x+h)-2u(x)+u(x-h)}{h^2}, \\
% 			u'(x)  & \approx   \frac{u(x+h)-u(x-h)}{2h},
% 		\end{aligned}
% 	}
% \end{align*}

% \newpage
% Derivation the system of equations

% \newpage
% Derivation the system of equations (cont.)

% \newpage
% \begin{enumerate}
% 	\item Discretization for $n = 1,..,N-1$
% 	      \begin{align*}
% 		      \frac{u_{n+1} - 2u_n + u_{n-1}}{h^2} -
% 		      \frac{u_{n+1}-u_{n-1}}{2h} + u_n 
% 		      = 2x_n - 1 - x_n^2,
% 	      \end{align*}
% 	      which leads to the following grouped form
% 	      \begin{align*}
% 		      \boxed{
% 			      (2+h)u_{n-1} + (2h^2-4)u_n + (2-h)u_{n+1}
% 			      = 2h^2 \left( 2nh - 1 - (nh)^2 \right).
% 		      }
% 	      \end{align*}

% 	\item Left boundary exists \textbf{ghost} point $u_{-1}$ taking the following expression
% 	      \begin{align*}
% 		      u'(0) 
% 		      = 0~~\Rightarrow~~\frac{u_1-u_{-1}}{2h} 
% 		      = 0~~\Rightarrow~~u_1 
% 		      = u_{-1}
% 	      \end{align*}
% 	      which yields the first equation for the case $n=0$ as follows
% 	      \begin{align*}
% 		      \boxed{
% 			      (2h^2-4)u_0 + 4 u_{1} = -2h^2.
% 		      }
% 	      \end{align*}

% 	\item The right boundary where $u(1)=0$, meaning $u_N=0$,
% 	      is for the case of $n=N-1$
% 	      \begin{align*}
% 		      \boxed{
% 			      (2+h)u_{N-2} + (2h^2-4)u_{N-1} = 2h^2(2(1-h)-1-(1-h)^2).
% 		      }
% 	      \end{align*}

% 	\item The final system of equations takes the following form
% 	      \begin{align*}
% 		      \therefore\quad\boxed{
% 		      A_h=
% 		      \begin{pmatrix}
% 			      2h^2-4 & 4      &        &        &        \\
% 			      2+h    & 2h^2-4 & 2-h    &        &        \\
% 			             & 2+h    & 2h^2-4 & 2-h    &        \\
% 			             &        & \ddots & \ddots & \ddots \\
% 			             &        &        & 2+h    & 2h^2-4
% 		      \end{pmatrix}_{N\times N},
% 		      }
% 	      \end{align*}

% 	      \begin{align*}
% 		      \therefore\quad\boxed{
% 		      u_h =
% 		      \begin{pmatrix}
% 			      u_0    \\
% 			      u_1    \\
% 			      u_2    \\
% 			      \vdots \\
% 			      u_{N-1}
% 		      \end{pmatrix}_{N \times 1},
% 		      \quad
% 		      \quad
% 		      b_h=2h^2
% 		      \begin{pmatrix}
% 			      -1          \\
% 			      2h-1-h^2    \\
% 			      4h-1-(2h)^2 \\
% 			      \vdots      \\
% 			      2 (N-1)h-1 - \left( (N-1)h \right)^2
% 		      \end{pmatrix}_{N \times 1}.
% 		      }
% 	      \end{align*}

% 	      % \item Da für die ersten beiden Ableitungen gem\"{a}{\ss}
% 	      %       Taylorentwicklung-Formel gilt, dass
% 	      %       \begin{align*}
% 	      % 	      u''_h(x) & =  \frac{u(x+h)-2u(x)+u(x-h)}{h^2} + \mathcal{O}(h^2)\cdot u^{(4)}(x), \\
% 	      % 	      %   u''_h(x) & = \frac{u(x+h)-2u(x)+u(x-h)}{h^2} + \mathcal{O}(h^2), \\
% 	      % 	      u'_h(x)  & =   \frac{u(x+h)-u(x-h)}{2h} + \mathcal{O}(h^2)\cdot u^{(3)}(x),       \\
% 	      % 	      %   u'_h(x)  & = \frac{u(x+h)-u(x-h)}{2h} + \mathcal{O}(h^2),        \\
% 	      % 	      u_h(x)   & = u(x),
% 	      %       \end{align*}

% 	      %       und weiterhin die exakte L\"{o}sung unendlich oft differenzierbar ist mit
% 	      %       \begin{align*}
% 	      % 	      u^{(n)} = 0,\ \forall\ n\in\mathbb{R},\ n>2,
% 	      %       \end{align*}
% 	      %       wird die L\"{o}sung mit diesem Finite Differenzen Schema exakt
% 	      %       approximiert, das hei{\ss}t die Differenz zwischen $u_h$ und $u$
% 	      %       ist $0$ in jedem Punkt. Sei
% 	      %       \begin{align*}
% 	      % 	      f(x)=u''(x) - u'(x) + u(x),
% 	      % 	      \text{ also }
% 	      % 	      f(x) := 2x-1-x^2.
% 	      %       \end{align*}

% 	      %       Für den Konsistenzfehler erhalten wir somit
% 	      %       \begin{align*}
% 	      % 	      \| \left(A_h u\right)|_{\Omega_h} - f|_{\Omega_h} \|_\infty
% 	      % 	       & = \max_{x\in\Omega_h}
% 	      % 	      \left|
% 	      % 	      \left( u''_h(x) - u'_h(x) + u_h(x) \right)
% 	      % 	      - \left( u''(x) - u'(x) + u(x)  \right)
% 	      % 	      \right|                                                     \\
% 	      % 	       & \leq \max_{x\in \Omega_h} \left| u''_h(x)-u''(x) \right|
% 	      % 	      + \max_{x \in \Omega_h} \left| u'_h(x)-u'(x) \right|
% 	      % 	      + \max_{x \in \Omega_h} \left| u_h(x)-u(x) \right|          \\
% 	      % 	       & \leq \mathcal{O}(h^2) \left| u^{(4)}(x) \right|_\infty
% 	      % 	      + \mathcal{O}(h^2) \left| u^{(3)}(x) \right|_\infty
% 	      % 	      + 0                                                         \\
% 	      % 	       & \leq 0.
% 	      %       \end{align*}

% 	      %       Jede Norm ist nicht negative und damit erhalten wir den Konsistenzfehler
% 	      %       \begin{align*}
% 	      % 	      \therefore\quad\boxed{
% 	      % 	      \| \left(A_h u\right)|_{\Omega_h} - f|_{\Omega_h} \|_\infty = 0.
% 	      % 	      }
% 	      %       \end{align*}
% \end{enumerate}

% %------------------------------------------------------------------------------
% \newpage
% \begin{example}
% 	Examine the following FDM problem:
% 	\begin{align*}
% 		u''(x) - u'(x) + u(x) = -x^2 + 2x - 1,
% 		\quad
% 		\text{for }\  x\in \Omega =(0,1),
% 	\end{align*}
% 	\inputfig{floats/discretizeGhostRight}{discretizeGhostRight}
% 	where the boundary conditions are defined as follows
% 	\begin{align*}
% 		u(0) = 0, \quad u'(1) = 0.
% 	\end{align*}
% 	Determine the resulting system of linear equations 
% 	$A_h u_h = b_h$ by using second order FDM scheme for both $u''(x)$ and $u'(x)$. 
% \end{example}

% Approach: This problem exists a ghost point $\mathrightghost$
% on the RHS boundary $\partial\Omega_{R}$.\\

% The second order FDM scheme for both $u''(x)$ and $u'(x)$ take the following forms
% \begin{align*}
% 	\boxed{
% 		\begin{aligned}
% 			u''(x) & \approx  \frac{u(x+h)-2u(x)+u(x-h)}{h^2}, \\
% 			u'(x)  & \approx   \frac{u(x+h)-u(x-h)}{2h},
% 		\end{aligned}
% 	}
% \end{align*}

% %------------------------------------------------------------------------------
% \newpage
% \section{Evaluation}

% \inputfig{floats/evaluationQR}{evaluationQR}

% Alternative link:
% % \url{https://www.campus.rwth-aachen.de/evasys/online.php?pswd=5F6HTY8K3C}
% \href{https://www.campus.rwth-aachen.de/evasys/online.php?pswd=5F6HTY8K3C}{here}

% \inputfig{floats/nodiscretize}{nodiscretize}
% Approach:
% \begin{enumerate}
% 	\item Homogenization of the BCs: We search for $u_{inh}(x,t)$ such that
% 	      \begin{align}
% 		      \label{eq:uinh}
% 		      \partial_{xx}u_{inh} = 0, \quad \text{with BCs:}\ 
% 		      \begin{cases}
% 			      u_{inh}(0,t) = 0,                  \\
% 			      u_{inh}(L,t) = \mu t \exp(-\mu t), \\
% 		      \end{cases}
% 	      \end{align}
% 	      which leads to
% 	      \begin{align}
% 		      \label{eq:uinhsol}
% 		      \boxed{
% 			      u_{inh}(x,t) = \frac{x}{L}\mu t \exp(-\mu t).
% 		      }
% 	      \end{align}
% 	      Likewise, we also obtain the following expression
% 	      \begin{align}
% 		      \label{eq:ptuinh}
% 		      \partial_{t}u_{inh} 
% 		      = \frac{x}{L}\mu \exp(-\mu t) - \frac{x}{L}\mu^{2} t \exp(-\mu t)
% 		      = \frac{x}{L}(1-\mu t)\mu \exp(-\mu t)
% 	      \end{align}
% 	      Note in passing that we have used $u_{inh}(x,t) = ax+b, \forall a,b \in\mathbb{R}$
% 	      to probe the solution mentioned at \eqref{eq:uinh}, since it satisfies $\partial_{xx}(ax+b) = 0$.
% 	      The solution \eqref{eq:uinhsol} is obtained by considering BCs given in \eqref{eq:uinh}.

% 	\item We define
% 	      \begin{align}
% 		      \label{eq:utilde}
% 		      \tilde{u}(x,t) := u(x,t) - u_{inh}(x,t) 
% 		      \Leftrightarrow 
% 		      u(x,t) = \tilde{u}(x,t) + u_{inh}(x,t)
% 	      \end{align}
% 	      By substituting \eqref{eq:utilde} into \eqref{eq:probleminh} we obtain the following equality
% 	      \begin{align}
% 		      \partial_t(\tilde{u} + u_{inh}) = \partial_{xx}(\tilde{u} + u_{inh})
% 		       & \Leftrightarrow
% 		      \partial_t\tilde{u} - \partial_{xx}\tilde{u}
% 		      = \cancelto{0; \text{cf.} \eqref{eq:uinh}}{\partial_{xx}u_{inh}} - \partial_{t}u_{inh} \notag \\
% 		       & \Leftrightarrow
% 		      \partial_t\tilde{u} - \partial_{xx}\tilde{u}
% 		      = - \partial_{t}u_{inh}                                                                \notag \\
% 		       & \Leftrightarrow
% 		      \partial_t\tilde{u} - \partial_{xx}\tilde{u}
% 		      \stackrel{\eqref{eq:ptuinh}}{=}
% 		      - \frac{x}{L}(1-\mu t)\mu \exp(-\mu t)
% 	      \end{align}
% 	      Then, the homogenized problem reads
% 	      \begin{equation}
% 		      \label{eq:utildeeqn}
% 		      \boxed{
% 			      \begin{aligned}
% 				      \partial_t\tilde{u}(x,t) - \partial_{xx}\tilde{u}(x,t) = \tilde{f}(x,t),
% 				      \quad \text{with}
% 				      \
% 				      \begin{cases}
% 					      \tilde{u}(0,t)=\tilde{u}(L,t)= 0, \\
% 					      \tilde{u}(x,0)= u_{0}\ x(L-x),
% 				      \end{cases}
% 			      \end{aligned}
% 		      }
% 	      \end{equation}
% 	      where the source term $\tilde{f}(x,t)$ defined as 
% 	      $\displaystyle \tilde{f}(x,t) := -\frac{x}{L}(1-\mu t)\mu \exp(-\mu)$.

% 	\item We now need to find out the Eigenfunctions $\varphi(x)$ satisfying
% 	      \begin{align}
% 		      \label{eq:eigenfunctionproblemlambda}
% 		      -\partial_{xx}\varphi(x) = \lambda\varphi(x),
% 		      \quad\text{with}\ \varphi(0)=\varphi(L)=0.
% 	      \end{align}
% 	      By taking into consideration the Ansatz
% 	      \begin{align}
% 		      \label{eq:ansatz}
% 		      \begin{cases}
% 			      \begin{aligned}
% 				      \varphi(x)   & = \exp(rx)      \\
% 				      \varphi''(x) & = r^{2}\exp(rx)
% 			      \end{aligned}
% 		      \end{cases}
% 	      \end{align}
% 	      We then obtain the following expressions
% 	      \begin{align}
% 		      \eqref{eq:eigenfunctionproblemlambda}
% 		      \stackrel{\eqref{eq:ansatz}}{\Leftrightarrow}
% 		      \exp(rx)(r^{2} + \lambda) = 0
% 		      \Leftrightarrow
% 		      r = \pm i\sqrt{\lambda}
% 	      \end{align}
% 	      The solution to \eqref{eq:eigenfunctionproblemlambda} $\forall A,B,C_{1},C_{2} \in \mathbb{R}$ takes the following form 
% 	      \begin{align}
% 		      \label{eq:varphix}
% 		      \varphi(x)
% 		       & = A\exp\left(i\sqrt{\lambda}x\right) + B\exp\left(-i\sqrt{\lambda}x\right) \notag          \\
% 		       & = A\left( \cos\left(\sqrt{\lambda}x\right) + i\sin\left(\sqrt{\lambda}x\right) \right)
% 		      + B\left( \cos\left(\sqrt{\lambda}x\right) - i\sin\left(\sqrt{\lambda}x\right) \right) \notag \\
% 		       & = (A+B)\cos\left(\sqrt{\lambda}x\right) + (A-B)i\sin\left(\sqrt{\lambda}x\right) \notag    \\
% 		       & = C_{1}\cos\left(\sqrt{\lambda}x\right) + iC_{2}\sin\left(\sqrt{\lambda}x\right)
% 	      \end{align}
% 	      Note in passing that Euler's formula $\exp(ix) = \cos(x) +i\sin(x)$ has been used in \eqref{eq:varphix}.
% 	      Next, by applying BCs from \eqref{eq:eigenfunctionproblemlambda}
% 	      we obtain the explicit values for $C_{1}$ and eigenvalues $\lambda_{k}$ for \eqref{eq:varphix}, as follows
% 	      \begin{align}
% 		       & \varphi(0)=0 \stackrel{\eqref{eq:varphix}}{\Leftrightarrow} C_{1} = 0. \label{eq:C1} \\
% 		       & \varphi(L)=0 \stackrel{\eqref{eq:varphix}\eqref{eq:C1}}{\Leftrightarrow}
% 		      iC_{2}\sin\left( \sqrt{\lambda}L \right) = 0
% 		      \Leftrightarrow \lambda_{k} = \left( \frac{k\pi}{L} \right)^2,\quad\forall \ k\in\mathbb{Z}.
% 	      \end{align}
% 	      Till this point we have obtained so far only \textbf{non-normalized}
% 	      eigenfunctions $\varphi_k(x)$ and their eigenvalues $\lambda_k$ as follows
% 	      \begin{align}
% 		      \label{eq:nonnomalized}
% 		      \begin{cases}
% 			      \begin{aligned}
% 				      \varphi_k(x) & = iC_{2} \sin\left(\frac{k \pi}{L} x\right), \\
% 				      \lambda_k    & = \left( \frac{k \pi}{L} \right)^2,
% 			      \end{aligned}
% 			      \quad\forall\ k\in\mathbb{Z}.
% 		      \end{cases}
% 	      \end{align}
% 	      Next, the eigenfunction $\varphi_k(x)$  in \eqref{eq:nonnomalized} needs to be normalized. It goes as follows
% 	      \begin{align}
% 		      ||\varphi_{k}(x)||_{2} \stackrel{!}{=} 1 
% 		       & \Leftrightarrow \sqrt{\int_{0}^{L} \left( \varphi_{k}(x) \right)^2\ dx } = 1 \notag   \\
% 		       & \stackrel{\eqref{eq:nonnomalized}}{\Leftrightarrow }
% 		      \int_{0}^{L} \left( iC_{2} \sin\left(\frac{k \pi}{L} x\right) \right)^2 \ dx  = 1 \notag \\
% 		       & \Leftrightarrow 
% 		      \int_{0}^{L} \sin^2\left(\frac{k \pi}{L} x\right) \ dx  = \frac{-1}{(C_{2})^2} \notag    \\
% 		       & \Leftrightarrow 
% 		      \int_{0}^{L} \left(\frac{1}{2} - \frac{1}{2}\cos\left( \frac{2k \pi}{L} x \right) \right) \ dx 
% 		      = \frac{-1}{(C_{2})^2} \notag                                                            \\
% 		       & \Leftrightarrow 
% 		      \frac{1}{2}x\Bigg|_{x=0}^{x=L} 
% 		      -\frac{1}{2}\frac{L}{2k\pi}\sin\left(\frac{2k\pi}{L}x\right)\Bigg|_{x=0}^{x=L}
% 		      = \frac{-1}{(C_{2})^2} \notag                                                            \\
% 		       & \Leftrightarrow
% 		      \frac{1}{2}L = \frac{-1}{(C_{2})^2}
% 		      \Leftrightarrow
% 		      C_{2} = \pm i\sqrt{\frac{2}{L}}
% 	      \end{align}
% 	      \begin{itemize}
% 		      \item $\displaystyle C_{2} = i\sqrt{\frac{2}{L}}
% 			            \stackrel{\eqref{eq:nonnomalized}}{\Rightarrow}
% 			            \varphi_k(x)
% 			            % = iC_{2} \sin\left(\frac{k \pi}{L} x\right)
% 			            = - \sqrt{\frac{2}{L}}\sin\left(\frac{k \pi}{L} x\right)
% 			            = \sqrt{\frac{2}{L}}\sin\left(\frac{h \pi}{L} x\right)
% 			            \forall\ k,h\in\mathbb{Z}.
% 		            $
% 		      \item $\displaystyle C_{2} = - i\sqrt{\frac{2}{L}}
% 			            \stackrel{\eqref{eq:nonnomalized}}{\Rightarrow}
% 			            \varphi_k(x)
% 			            % = iC_{2} \sin\left(\frac{k \pi}{L} x\right)
% 			            = + \sqrt{\frac{2}{L}}\sin\left(\frac{k \pi}{L} x\right)
% 			            = \sqrt{\frac{2}{L}}\sin\left(\frac{k \pi}{L} x\right)
% 			            \forall\ k\in\mathbb{Z}.
% 		            $
% 	      \end{itemize}

% 	      Therefore, the \textbf{normalized} eigenfunctions $\varphi_k(x)$ and their eigenvalues $\lambda_k$
% 	      take the following forms
% 	      \begin{equation*}
% 		      \therefore\quad
% 		      \boxed{
% 			      \begin{aligned}
% 				      \varphi_k(x) & = \sqrt{\frac{2}{L}} \sin\left(\frac{k \pi}{L} x\right) \\
% 				      \lambda_k    & = \left( \frac{k \pi}{L} \right)^2, 
% 				      \forall\ k\in\mathbb{Z}
% 			      \end{aligned}
% 		      }
% 	      \end{equation*}

% 	\item Expand the source term $\tilde{f}(x,t)$ by using the normalized eigenfunction $\varphi_{k}(x)$.
% 	      The source term $\tilde{f}(x,t)$ takes the following form
% 	      \begin{align}
% 		      \tilde{f}(x,t) = -\frac{x}{L} (1-\mu t)\mu \exp(-\mu t),
% 	      \end{align}
% 	      which can be written in an expansion of eigenfunctions
% 	      \begin{align}
% 		      \label{eq:ftilde}
% 		      \boxed{
% 			      \tilde{f}(x,t)
% 			      = \sum_{k=1}^\infty \beta_k(t) \varphi_{k}(x)
% 			      = \sum_{k=1}^\infty \beta_k(t) \sqrt{\frac{2}{L}} \sin\left( \frac{k \pi}{L} x \right)
% 		      }
% 	      \end{align}
% 	      where coefficient $\beta_{k}(t)$ is computed by projecting $\tilde{f}(x,t)$ on eigenfunctions $\varphi_{k}(x)$, 
% 	      as follows
% 	      \begin{align}
% 		      \label{eq:betacoefficient}
% 		      \beta_k(t)
% 		       & = \langle \tilde{f}(x,t),\varphi_{k}(x) \rangle             \notag \\
% 		       & = \int_0^L 
% 		      \left(-\frac{x}{L} (1-\mu t)\mu \exp(-\mu t) \right)
% 		      \sqrt{\frac{2}{L}} \sin\left(\frac{k \pi}{L} x\right) \ dx     \notag \\
% 		       & = \frac{-(1-\mu t) \mu \exp(-\mu t)}{L} \sqrt{\frac{2}{L}}
% 		      \int_0^L x
% 		      \sin\left(\frac{k \pi}{L} x\right) \ dx                        \notag \\  
% 		       & = \frac{-(1-\mu t) \mu \exp(-\mu t)}{L} \sqrt{\frac{2}{L}}
% 		      \frac{(-1)^k}{k\pi}L^2                                         \notag \\
% 		       & = (-1)^k \frac{\sqrt{2L}}{k \pi}(1-\mu t)\mu \exp(-\mu t).
% 	      \end{align}

% 	\item Ansatz for solution $\tilde{u}(x,t)$ take the following form
% 	      \begin{align}
% 		      \label{eq:ansatzu}
% 		      \tilde{u}(x,t) = \sum_{k=1}^\infty \alpha_k(t) \varphi_k(x).
% 	      \end{align}
% 	      By substituting \eqref{eq:ftilde} and \eqref{eq:ansatzu} into \eqref{eq:utildeeqn} we obtain 
% 	      \begin{align}
% 		      \label{eq:ODEalpha}
% 		       & \Leftrightarrow 
% 		      \sum_{k=1}^\infty 
% 		      \left(
% 		      \partial_t \alpha_{k}(t) \varphi_k(x)
% 		      - \alpha_k(t) \partial_{xx} \varphi_k(x)
% 		      - \beta_k(t) \varphi_k(x)
% 		      \right)
% 		      = 0             \notag                                               \\
% 		       & \stackrel{\eqref{eq:eigenfunctionproblemlambda}}{\Leftrightarrow}
% 		      \sum_{k=1}^\infty 
% 		      \left( 
% 		      \alpha_{k}'(t) \varphi_k(x)
% 		      + \alpha_k(t) \lambda_k \varphi_k(x)
% 		      - \beta_k(t) \varphi_k(x)
% 		      \right)              
% 		      = 0             \notag                                               \\
% 		       & \Leftrightarrow 
% 		      \sum_{k=1}^\infty 
% 		      \left( \alpha_{k}'(t)  + \lambda_k\alpha_k(t) - \beta_k(t) \right) \varphi_k(x)
% 		      = 0.
% 	      \end{align}
% 	      Since the eigenfunctions $\varphi_{k}(x)$ are orthonormal to each other, 
% 	      we induce from \eqref{eq:ODEalpha} and arrive at the following equality 
% 	      \begin{align}
% 		      \label{eq:ODEalphafinal}
% 		      \boxed{
% 			      \alpha_{k}'(t) + \lambda_k \alpha_k(t) - \beta_k(t) = 0.
% 		      }
% 	      \end{align}
% 	      This is the 1st order ODE with unknown $\alpha_{k}(t)$. 
% 	      The solution to $\alpha_k(t)$ takes the form
% 	      \begin{align*}
% 		      \therefore\quad
% 		      \boxed{
% 			      \alpha_k(t)
% 			      = \alpha_k(0) \exp(-\lambda_k t)
% 			      + \exp(-\lambda_k t) \int_0^t \exp(\lambda_k \tau) \beta_k(\tau) \ d\tau,
% 		      }
% 	      \end{align*}
% 	      whose derivation goes as follows
% 	      \begin{itemize}
% 		      \item Solving for homogeneous part:
% 		            \begin{align}
% 			            \alpha_{k}'(t) + \lambda_k \alpha_k(t) = 0 
% 			            \Leftrightarrow
% 			            \alpha^{hom}_{k}(t) = \alpha_{k}(0)\exp(-\lambda_k t).
% 		            \end{align}
% 		      \item Inhomogeneous part: By using the method \emph{variation of constant}
% 		            we exploit the ansatz 
% 		            \begin{align}
% 			            \label{eq:ansatzforalpha}
% 			            \alpha^{inh}_{k}(t) = \alpha_{k}(t)\exp(-\lambda_k t).
% 		            \end{align}
% 		            Insertion of \eqref{eq:ansatzforalpha} into \eqref{eq:ODEalphafinal} leads to 
% 		            \begin{align*}
% 			            \alpha'_{k}(t)\exp(-\lambda_k t)
% 			            - \lambda_k \alpha_{k}(t)\exp(-\lambda_k t)
% 			            + \lambda_k \alpha_{k}(t)\exp(-\lambda_k t)
% 			            - \beta_{k}(t)                   = 0                  
% 		            \end{align*}
% 		            where the two terms in the middle have been canceled out with each other. 
% 		            We then obtain the equality
% 		            \begin{align}
% 			            \alpha'_{k}(t)\exp(-\lambda_k t)
% 			             & = \beta_{k}(t) \notag                                                                              \\
% 			            \Leftrightarrow \alpha'_{k}(t)   
% 			             & = \exp(\lambda_k t)\beta_{k}(t)                                                      \notag        \\
% 			            \Leftrightarrow \alpha_{k}(t)
% 			             & = \int_{0}^{t} \exp(\lambda_k \tau)\beta_{k}(\tau)\ d\tau                            \notag        \\
% 			            \Leftrightarrow \alpha_{k}(t)
% 			             & \stackrel{\eqref{eq:betacoefficient}}{=}
% 			            \int_0^t \exp(\lambda_k \tau)
% 			            (-1)^k \frac{\sqrt{2 L}}{k \pi} (1- \mu\tau) \mu \exp(-\mu\tau) \ d\tau                       \notag  \\
% 			            \Leftrightarrow \alpha_{k}(t)
% 			             & = (-1)^k \frac{\sqrt{2L}}{k \pi} \mu \int_0^t 
% 			            \exp(\lambda_{k}\tau) ( 1 - \mu\tau) \exp(-\mu\tau) \ d\tau                                    \notag \\
% 			            \Leftrightarrow \alpha_{k}(t)
% 			             & = (-1)^k \frac{\sqrt{2L}}{k \pi} \mu 
% 			            \left(
% 			            \int_0^t \exp(\lambda_k \tau - \mu\tau)  \ d\tau
% 			            - \mu \int_0^t \tau \exp(\lambda_k \tau -\mu\tau ) \ d\tau 
% 			            \right)                                                                                 \notag        \\
% 			            \Leftrightarrow \alpha_{k}(t)
% 			             & = (-1)^k \frac{\sqrt{2L}}{k \pi} \mu  
% 			            \frac{(\lambda_k - (\lambda_k-\mu)\mu t) \exp(\lambda_k -\mu t) - \lambda_k}
% 			            {(\lambda_k -\mu)^2}  \notag                                                                          \\
% 			            \Leftrightarrow \alpha_{k}(t)
% 			             & = (-1)^k \frac{\sqrt{2L}}{k \pi} \mu  
% 			            \frac{
% 				            (\lambda_k - (\lambda_k-\mu)\mu t) \exp(-\mu t)
% 				            - \lambda_k \exp(-\lambda_k t)}
% 			            {(\lambda_k -\mu)^2}
% 		            \end{align}
% 	      \end{itemize}

% 	\item Find $\alpha_k(0)$ by taking into consideration the IC of $\tilde{u}(x,0)$, as follows
% 	      \begin{align}
% 		      \tilde{u}(x,0) = \sum_{k=1}^{\infty} \alpha_k(0) \varphi_k(x)
% 		      \stackrel{\eqref{eq:utildeeqn}}{\Leftrightarrow}
% 		      u_0\ x(L-x) \stackrel{!}{=} \sum_{k=1}^{\infty} \alpha_k(0) \varphi_k(x)
% 	      \end{align}
% 	      where the coefficient $\alpha_k(0)$ is computed as follows
% 	      \begin{align}
% 		      \alpha_k(0)
% 		       & = \langle \tilde{u}(x,0),\varphi_k(x) \rangle                                           \notag \\
% 		       & = \int_0^L u_0\ x (L-x) \sqrt{\frac{2}{L}} \sin\left( \frac{k \pi}{L} x \right) \ dx 
% 		      =
% 		      \begin{cases}
% 			      \displaystyle
% 			      u_0 \sqrt{2L} \frac{4 L^2}{(k\pi)^3}, & k \text{ odd}, \\
% 			      0,                                    & k\text{ even}.
% 		      \end{cases}
% 	      \end{align}
% 	      The solution to $\tilde{u}(x,t)$ then reads
% 	      \begin{equation}
% 		      \label{eq:utildesolution}
% 		      %   \therefore\quad
% 		      \boxed{
% 			      \begin{aligned}
% 				      \tilde{u}(x,t)
% 				       & = \sum_{k=1}^\infty \alpha_k(t) \varphi_k(x)  \\
% 				       & = \sum_{\substack{k=1                         \\k \text{ odd}}}^\infty \frac{8 u_0 L^2}{(k\pi)^3}
% 				      \exp(-\lambda_k t)
% 				      \sin\left( \frac{k \pi}{L} x \right)             \\
% 				       & \ + \sum_{k=1}^\infty (-1)^k \frac{2\mu}{k \pi}
% 				      \frac{(\lambda_k - (\lambda_k-\mu)\mu t)\exp(-\mu t) - \lambda_k \exp(-\lambda_k t)}
% 				      {(\lambda_k - \mu)^2}
% 				      \sin\left( \frac{k \pi}{L} x \right)
% 			      \end{aligned}
% 		      }
% 	      \end{equation}

% 	\item Finally, by substituting \eqref{eq:ptuinh} and \eqref{eq:utildesolution}
% 	      into  \eqref{eq:utilde}
% 	      the solution to $u(x,t)$ reads
% 	      \begin{align*}
% 		      \therefore\quad\boxed{
% 			      u(x,t) = \tilde{u}(x,t) + \frac{x}{L} \mu t\exp(-\mu t).
% 		      }
% 	      \end{align*}
% \end{enumerate}

% Similarly to steps approaching to \eqref{eq:ODEalphafinal}, which is shown in Example \ref{ex:example1}, 
% we obtain a 2nd order ODE, instead of the 1st order ODE, as follows
% \begin{align}
% 	\label{eq:2ndODEform}
% 	\therefore\quad
% 	\boxed{
% 		\alpha_{k}''(t) + \lambda_k \alpha_k(t) - \beta_k = 0.
% 	}
% \end{align}
% Essentially, the equation \eqref{eq:2ndODEform} has the form 
% \begin{align}
% 	y''(x) = -cy(x) + b,\quad\forall \ c,b\in\mathbb{R},
% \end{align}
% whose solution takes the following form
% \begin{align}
% 	y(x) = \frac{b}{c}
% 	+ C_{1}\sin\left(\sqrt{c}x\right)
% 	+ C_{2}\cos\left(\sqrt{c}x\right),\quad\forall \ C_{1},C_{2}\in\mathbb{R}.
% \end{align}


% Approach: By using Taylor expansion for $f(x+h)$ and $f(x+2h)$
% we obtain the following expressions
% \begin{align}
% 	\label{eq:firstfhreeTaylor}
% 	\begin{cases}
% 		\begin{aligned}
% 			f(x)    & = f(x)   \\
% 			f(x+ h) & = f(x) 
% 			+ \frac{h^1}{1!} f'(x)  
% 			+ \frac{h^2}{2!} f''(x)  
% 			+ \frac{h^3}{3!} f'''(x) 
% 			+ \mathcal{O}(h^4) \\
% 			f(x+2h) & = f(x) 
% 			+ \frac{(2h)^1}{1!} f'(x) 
% 			+ \frac{(2h)^2}{2!} f''(x) 
% 			+ \frac{(2h)^3}{3!} f'''(x) 
% 			+ \mathcal{O}(h^4)
% 		\end{aligned}
% 	\end{cases}
% \end{align}
% which leads to 
% \begin{align}
% 	\alpha f(x)    & = \alpha f(x)  \label{eq:alphafx} \\
% 	\beta f(x+ h)  & = \beta f(x) 
% 	+ \beta h f'(x)  
% 	+ \frac{\beta h^2}{2} f''(x)
% 	+ \frac{\beta h^3}{6} f'''(x)
% 	+ \mathcal{O}(h^4)              \label{eq:betafxh} \\
% 	\gamma f(x+2h) & = \gamma f(x) 
% 	+ 2\gamma h f'(x)  
% 	+ 2\gamma h^2 f''(x)
% 	+ \frac{4\gamma h^3}{3} f'''(x)
% 	+ \mathcal{O}(h^4) \label{eq:gammafx2h}
% \end{align}
% Summation of \eqref{eq:alphafx}, \eqref{eq:betafxh} and \eqref{eq:gammafx2h} leads to
% \begin{align}\label{eq:summationall}
% 	\alpha f(x) + \beta f(x+h) + \gamma f(x+2h) 
% 	 & = (\alpha + \beta + \gamma)f(x) \notag                                      \\
% 	 & \quad + (\beta h + 2 \gamma h)f'(x) \notag                                  \\
% 	 & \quad + \left( \frac{\beta h^2}{2} + 2\gamma h^2 \right)f''(x) \notag       \\
% 	 & \quad + \left( \frac{\beta h^3}{6} + \frac{4\gamma h^3}{3}  \right) f'''(x)
% 	+ \mathcal{O}(h^4)
% \end{align}
% Comparison of \eqref{eq:summationall} with \eqref{eq:givenproblemfxprime}
% yields the following equalities
% \begin{align}
% 	\begin{cases}
% 		\alpha + \beta + \gamma = 0 \\
% 		\beta h + 2 \gamma h =  1   \\
% 		\displaystyle \frac{\beta h^2}{2} + 2\gamma h^2 = 0
% 	\end{cases}
% 	\Leftrightarrow
% 	\begin{pmatrix}
% 		1 & 1   & 1 \\
% 		0 & 1   & 2 \\
% 		0 & 1/2 & 2
% 	\end{pmatrix}
% 	\begin{pmatrix}
% 		\alpha \\ \beta \\ \gamma
% 	\end{pmatrix}
% 	=
% 	\begin{pmatrix}
% 		0 \\ 1/h \\ 0
% 	\end{pmatrix}
% 	\Leftrightarrow
% 	\begin{pmatrix}
% 		\alpha \\ \beta \\ \gamma
% 	\end{pmatrix}
% 	=
% 	\begin{pmatrix}
% 		-3/(2h) \\ 2/h \\ -1/(2h)
% 	\end{pmatrix}
% \end{align}
% which leads to
% \begin{align}
% 	\label{eq:finalsum}
% 	af(x) + bf(x+h) + cf(x+2h) 
% 	\approx
% 	f'(x) 
% 	+ f'''(x) \underbrace{\left( \frac{2}{h}\frac{h^3}{6}+\frac{-1}{2h}\frac{4h^3}{3}\right)}
% 	_{\displaystyle =-h^2/3}
% 	+ \mathcal{O}(h^4).
% \end{align}
% Since the term $-h^2/3$ from \eqref{eq:finalsum} is nonzero,
% we only need to expand those two Taylor expansions from \eqref{eq:firstfhreeTaylor}
% till $\mathcal{O}(h^3)$. Therefore, the consistency order, 
% due to the factor $h$ coming along with $f'(x)$, i.e. $\mathcal{O}(h^3)/h$, 
% is guaranteed till at least order 2.

% \newpage
% (Example 4 cont.)

% Die Gleichung $y''(x) = -cy(x) +b$ besitzt die allg. Lösung 
% 	\begin{align*}
% 		y(x) = \frac{b}{c} + k_1 \sin(\sqrt{c} x)+ k_2\cos(\sqrt{c} x), \quad k_1,k_2 \in \mathbb{R}
% 	\end{align*}

% \begin{enumerate}
% 	\item Die zum Problem geh"orenden Eigenfunktionen und Eigenwerte erf"ullen
% 	      \begin{align*}
% 		      \varphi^{\prime\prime}(x)+\lambda\varphi(x) & =0, \quad x\in (0,1) \\
% 		      \varphi(0)=\varphi(1)                       & =0
% 	      \end{align*}
% 	      \begin{align*}
% 		      \Rightarrow \quad \varphi(x) = C_1\sin(\sqrt{\lambda}x) + C_2 \cos(\sqrt{\lambda}x)
% 	      \end{align*}
% 	      Mit den Randbedingungen f"ur $\varphi$ erh"alt man
% 	      \begin{align*}
% 		      0 = \varphi(0) & = C_2 \, \Rightarrow \, C_2 = 0                                                       \\
% 		      0 = \varphi(1) & = C_1\sin(\sqrt{\lambda}x) \, \Rightarrow \, \lambda_k = (k\pi)^2, \, k\in \mathbb{N}
% 	      \end{align*}
% 	      \begin{align*}
% 		      \Rightarrow \varphi_k(x) = C_1\sin(k\pi x) , \, k \in \mathbb{N}
% 	      \end{align*}
% 	      $C_1$ is obtained by taking into the consideration of normalization
% 	      \begin{align*}
% 		      1 & \stackrel{!}{=} \int_0^1 \varphi_k(x)^2\, dx = C_1^2 \int_0^1 \sin^2(k\pi x) \, dx                \\
% 		        & = C_1^2 \left[ \left. \frac x 2 - \frac{\sin(2k\pi x)}{4k\pi}\right|^1_0 \right] = C_1^2\frac 1 2 \\
% 		      \Rightarrow C_1 = \sqrt{2}
% 	      \end{align*}
% 	      \reversemarginpar
% 	      damit lauten die Eigenwerte und normierten Eigenfunktionen
% 	      \begin{align*}
% 		      \varphi_k(x)               & =\sqrt{2}\sin(k\pi x)                  \\
% 		      \text{und} \quad \lambda_k & =(k\pi)^2 \text{ f"ur } k\in\mathbb{N}
% 	      \end{align*}

% 	\item Die Entwicklung des konstanten Quellterms $a(x,t)=a$  in Eigenfunktionen lautet
% 	      \begin{align*}
% 		      a & = \sum^{\infty}_{k=1}\beta_k\varphi_k(x)         \\
% 		        & = \sum^{\infty}_{k=1}\beta_k\sqrt{2}\sin(k\pi x)
% 	      \end{align*}
% 	      Die Koeffizienten berechnet man durch
% 	      \begin{align*}
% 		      \beta_k & =\int^1_0 a\sqrt{2}\sin(k\pi x)dx                                         \\
% 		              & = a \sqrt{2} \left[ \left. -\frac{\cos(k \pi x)}{k\pi}\right|^1_0 \right] \\
% 		              & =\frac{a\sqrt{2}}{k\pi}(1-\cos(k\pi))=\begin{cases}\frac{2a\sqrt{2}}{k\pi} & , k\text{ ungerade} \\
% 			      0                       & , \text{sonst}
% 		      \end{cases}
% 	      \end{align*}
% 	      Für die Entwicklung der Anfangsbedingung $u(x,0) = x$ gilt:
% 	      \begin{align*}
% 		      x & = \sum^{\infty}_{k=1}\gamma_k\varphi_k(x)         \\
% 		        & = \sum^{\infty}_{k=1}\gamma_k\sqrt{2}\sin(k\pi x)
% 	      \end{align*}
% 	      mit
% 	      \begin{align*}
% 		      \gamma_k & = \int_{0}^{1} x \sqrt{2} \sin(k\pi x)dx                                                                               \\
% 		               & = \sqrt{2} \left[ - x \cos(k \pi x) \frac{1}{k \pi}  \right]_0^1 + \sqrt{2}  \int_0^1 \frac{1}{k \pi} \cos(k \pi x) dx \\
% 		               & =\frac{\sqrt{2}}{k \pi}  \cos(k \pi)+ \frac{\sqrt{2}}{k \pi} \left[ \frac{1}{k \pi} \sin(k \pi x) \right]_0^1          \\
% 		               & = (-1)^k \frac{\sqrt{2}}{k \pi}
% 	      \end{align*}

% 	\item F"ur die L"osung $u$ wird der Eigenfunktionen-Ansatz
% 	      \begin{align*}
% 		      u(x,t)=\sum^{\infty}_{k=1}\alpha_k(t)\varphi_k(x)
% 	      \end{align*}
% 	      gew"ahlt, mit zeitabh"angigen Koeffizienten $\alpha_k(t), \, t>0, k\in \mathbb{N}$. \\
% 	      Einsetzen der Entwicklungen von $u(x,t)$ und $a$ in die Differentialgleichung liefert
% 	      \begin{align*}
% 		                           & (\sum^{\infty}_{k=1}\alpha_k(t)\varphi_k(x))_{tt}=  (\sum^{\infty}_{k=1}\alpha_k(t)\varphi_k(x))_{xx}+\sum^{\infty}_{k=1}\beta_k\varphi_k(x) \\
% 		      \Leftrightarrow\quad & \sum^{\infty}_{k=1}\alpha''_k(t)\varphi_k(x)=-\sum^{\infty}_{k=1}\alpha_k(t)\lambda_k\varphi_k(x)+\sum^{\infty}_{k=1}\beta_k\varphi_k(x)     \\
% 		      \Leftrightarrow\quad & \sum^{\infty}_{k=1}(\alpha''_k(t)+\lambda_k\alpha_k(t)-\beta_k)\varphi_k(x)=0
% 	      \end{align*}
% 	      \normalmarginpar
% 	      Wir erhalten eine gew"ohnliche Differentialgleichung f"ur die Koeffizienten $\alpha_k(t)$
% 	      \begin{align*}
% 		      \alpha''_k(t) & =-\lambda_k\alpha_k(t)+\beta_k
% 	      \end{align*}
% 	      mit der allgemeinen L"osung (Hinweis)
% 	      \begin{align*}
% 		      \alpha_k(t) & =\frac{\beta_k }{\lambda_k}+ C_{k,1} \sin(\sqrt{\lambda_k} t) + C_{k,2} \cos(\sqrt{\lambda_k} t) .
% 	      \end{align*}
% 	      Die Konstanten werden durch die Anfangsbedingungen bestimmt. Dazu gilt zunächst wegen $u_t(x,0) = 0$:
% 	      \begin{align*}
% 		      u_t(x,0) = \sum_{k = 1}^{\infty} \alpha'_k(0) \sqrt{2} \sin(k\pi x) \stackrel{!}{=} 0                            \\
% 		      \Rightarrow \alpha'_k(0) = \sqrt{\lambda_k} C_{k,1} \cos(0) - \sqrt{\lambda_k} C_{k,2} \sin(0) \stackrel{!}{=} 0 \\
% 		      \Rightarrow C_{k,1} = 0
% 	      \end{align*}
% 	      Die Konstante $C_{k,2}$ wird durch die Anfangsbedingung $u(x,0) = x$ bestimmt. Dazu betrachte
% 	      \begin{align*}
% 		      u(x,0)=\sum^{\infty}_{k=1}\alpha_k(0)\sqrt{2}\sin(k\pi x)\stackrel{!}{=} \sum^{\infty}_{k=1}\gamma_k\sqrt{2}\sin(k\pi x) \\
% 		      \Rightarrow \alpha_k(0)\stackrel{!}{=} \gamma_k
% 	      \end{align*}
% 	      Also gilt
% 	      \begin{align*}
% 		      \alpha_k(0)         & =  \frac{\beta_k}{\lambda_k}+C_{k,2} \stackrel{!}{=} \gamma_k =(-1)^k \frac{\sqrt{2}}{k \pi} \\
% 		      \Rightarrow C_{k,2} & = \begin{cases} - \frac{2 a \sqrt{2} }{(k \pi)^3} - \frac{\sqrt{2}}{ k \pi}, k \text{ ungerade} \\
% 			      \frac{\sqrt{2}}{ k \pi},  \text{ sonst}
% 		      \end{cases}
% 	      \end{align*}
% 	      Die L"osung $u$ lautet daher insgesamt
% 	      \begin{align*}
% 		      u(x,t) & = \sum_{k=1}^{\infty} \left( \frac{\beta_k }{\lambda_k}+ C_{k,1} \sin(\sqrt{\lambda_k} t) + C_{k,2} \cos(\sqrt{\lambda_k} t) \right) \sqrt{2} \sin(k \pi x)                        \\
% 		             & = \sum_{k=1 \atop k \text{ gerade}}^{\infty} \left[  \frac{4a }{(k \pi)^3}  +\left( \frac{{2}}{ k \pi} \right)   \cos(k \pi t)  \right]  \sin(k \pi x)                             \\
% 		             & +  \sum_{k=1 \atop k \text{ ungerade}}^{\infty} \left[  \frac{4a }{(k \pi)^3}  - \left( \frac{{2}a }{(k \pi)^3} + \frac{{2}}{ k \pi} \right) \cos(k \pi t)  \right]  \sin(k \pi x)
% 	      \end{align*}
% \end{enumerate}

% \section{Conservation laws - Idea}
% % Integral form
% % \begin{equation}
% % 	\frac{d}{dt}\int_{\Omega}u\,d\Omega 
% % 	= -\int_{\partial\Omega}f(u)\cdot n\,d\partial\Omega
% % \end{equation}
% % Differential form
% % \begin{equation}
% % 	u_{t} + \nabla\cdot f(u) = 0 
% % \end{equation}
% \emph{Master balance principle}: due to the fact that the mathematical structure of the fundamental balance relations,
% namely of mass, linear momentum, moment of momentum, energy and entropy, is in principle identical, they can be 
% formulated within the concise shape of a master balance. To begin with, let $\Psi$ and $\BPsi$ be volume specific
% scalar and vector value densities of a physical quantity to be balanced, respectively. Then, the general balance relations take 
% the global \textbf{integral form} as follows 
% \begin{align}
% 	\frac{d}{dt}\int_{\Omega} \Psi \,d\Omega  & = 
% 	-\int_{\partial\Omega}\Bphi\cdot\Bn\,d\partial\Omega 
% 	+ \int_{\Omega}\sigma\,d\Omega
% 	+ \int_{\Omega}\hat{\Psi}\,d\Omega             \\
% 	\frac{d}{dt}\int_{\Omega} \BPsi \,d\Omega & = 
% 	-\int_{\partial\Omega}\BPhi\cdot\Bn\,d\partial\Omega 
% 	+ \int_{\Omega}\Bsigma\,d\Omega
% 	+ \int_{\Omega}\hat{\BPsi}\,d\Omega
% \end{align}
% where $(\Bphi\cdot\Bn)$ and $(\BPhi\cdot\Bn)$ describe action at the vicinity; $\sigma$ and 
% $\Bsigma$ present for action from a distance; and $\hat{\Psi}$ and $\hat{\BPsi}$ for production or nucleation.
% \begin{example}
% 	Conservation of mass: A derivation from integral to differential form.
% \end{example}
% \inputfig{floats/massx1x2}{massx1x2}
% Let $x$ represent the distance along the tube and let $\rho(x,t)$ be the mass density of the 
% fluid at point $x$ and time $t$. Then, the total mass $M$ in the section 
% $[x_{1},x_{2}]$ at time $t$ is defined as follows
% \begin{equation}
% 	M := \int_{x_{1}}^{x_{2}} \rho(x,t)\,dx.
% \end{equation}
% Assume that the walls of the tube are impermeable and mass is neither created nor destroyed, 
% this total mass $M$ in this interval $[x_1,x_2]$ varies over time only due to the fluid flowing
% across at the boundaries $x=x_{1}$ and $x=x_{2}$. 
% % Besides, the flow rate (also called \emph{rate of flow}, or \emph{flux}) is denoted as $f(\rho)$.
% Now let $v(x,t)$ be the measured velocity of flow at point $x$ and time $t$.
% Then, the flow rate (also called \emph{rate of flow}, or \emph{flux})
% passing the end points of the interval $[x_{1},x_{2}]$ at time $t$ is given by
% \begin{equation}
% 	\rho(x_{1},t)v(x_{1},t)-\rho(x_{2},t)v(x_{2},t).
% \end{equation}
% The rate of change of mass in $[x_{1},x_{2}]$ given by the difference of fluxes at 
% $x_1$ and $x_2$ flow rate has to be equal with the time rate of change of total mass
% \begin{equation}\label{eq:relation0}
% 	\boxed{
% 		\frac{d}{dt}\int_{x_{1}}^{x_{2}} \rho(x,t)\,dx 
% 		% \stackrel{!}{=} f(\rho(x_{1},t))-f(\rho(x_{2},t))
% 		= \rho(x_{1},t)v(x_{1},t)-\rho(x_{2},t)v(x_{2},t)
% 	}
% \end{equation}
% which is the \textbf{integral form} of the conservation law. Next, note that the following relation holds
% \begin{equation}\label{eq:relation1}
% 	\int_{x_{1}}^{x_{2}} \frac{\partial}{\partial x}f(\rho)\,dx 
% 	= f(\rho(x_{2},t))-f(\rho(x_{1},t)).
% \end{equation}
% Substitution of \eqref{eq:relation1} into \eqref{eq:relation0} leads to
% \begin{equation}\label{eq:relation2}
% 	\frac{d}{dt}\int_{x_{1}}^{x_{2}} \rho(x,t)\,dx 
% 	= -\int_{x_{1}}^{x_{2}}\frac{\partial}{\partial x}f(\rho)\,dx.
% \end{equation}
% If $\rho(x,t)$ is sufficiently smooth, time derivative on the LHS of \eqref{eq:relation2}
% can be brought inside the integral. Then, an arrangement of \eqref{eq:relation2} yields
% \begin{equation}\label{eq:relation3}
% 	\int_{x_{1}}^{x_{2}}
% 	\left(\frac{\partial}{\partial t}\rho(x,t) + \frac{\partial}{\partial x}f(\rho)\right)\,dx 
% 	= 0.
% \end{equation}
% Since \eqref{eq:relation3} holds for any choice of $x_{1}$ and $x_{2}$, the integrant must be vanished
% everywhere. Therefore, one obtains the \textbf{differential form} of the conservation law
% \begin{equation}\label{eq:deri}
% 	\therefore \quad
% 	\boxed{\rho_{t} + f(\rho)_{x} = 0}
% \end{equation}
% % \begin{enumerate}
% % 	\item gas dynamics
% % 	\item oil-water mixtures
% % 	\item plasmas
% % 	\item shallow water flows
% % 	\item meteology
% % 	\item traffic
% % 	\item ...
% % \end{enumerate}
% % \begin{definition}
% % 	Cauchy problem
% % \end{definition}
% % Cauchy problem\\
% % Riemann problem\\
% % \begin{equation}
% % 	u_{t} + f(u)_{x} = 0 
% % \end{equation}
% \begin{example}
% 	Different definitions of flux function yields different applications
% \end{example}
% \begin{enumerate}
% 	\item Advection equation: \eqref{eq:deri} with flux function
% 	      $\displaystyle  f(\rho) =  a\rho, \ a\in \mathbb{R^+}$, taking form
% 	      \begin{equation}
% 		      \boxed{
% 			      \rho_{t} + a\rho_{x} = 0
% 		      }
% 	      \end{equation}
% 	\item \emph{Inviscid} Burgers' equation: \eqref{eq:deri} with flux function
% 	      $\displaystyle  f(\rho) =  \frac{1}{2}\rho^2$, yielding
% 	      \begin{equation}
% 		      \boxed{
% 			      \rho_{t} + \rho\rho_{x} = 0
% 		      }
% 	      \end{equation}
% 	      which is used to illustrate the distortion of waveform in simple waves.
% 	      Meanwhile, \emph{viscous} Burgers' equation is a scalar parabolic PDE, taking the form
% 	      $$\rho_t + \rho\rho_{x} = \varepsilon\rho_{xx}$$
% 	      which is the simplest differential model for a fluid flow.
% 	\item Lighthill-Whitham-Richards (LWR) equation: \eqref{eq:deri} with flux function
% 	      $$\displaystyle  f(\rho) =  u_{max}\,\rho\left(1-\frac{\rho}{\rho_{max}}\right)$$
% 	      which is used to model traffic flow; $u_{max}$ is the given maximal speed of vehicle to be allowed, 
% 	      $\rho_{max}$ the given maximal density of vehicle; taking the form
% 	      \begin{equation}
% 		      \rho_{t} + \left(u_{max}-\frac{2u_{max}}{\rho_{max}}\rho\right)\rho_{x} = 0
% 	      \end{equation}
% 	\item (Typical) Buckley-Leverett petroleum equation: \eqref{eq:deri} with flux function
% 	      $$\displaystyle f(\rho) =  \frac{\rho^2}{\rho^2 + (1-\rho)^2\mu_{\text{water}}/\mu_{\text{oil}}}$$
% 	      which is a one-dimensional model for a two-phase flow; $\rho$ stands for water saturation
% 	      and takes the value in the interval $[0,1]$.
% 	\item \emph{Euler} equations of gas dynamics in 1D
% 	      %   \begin{align}
% 	      %       \begin{cases}
% 	      % 	      \begin{aligned}
% 	      % 		      \rho_{t}     & + (\rho v)_x & = 0 \\
% 	      % 		      (\rho v)_{t} & + (\rho v^2 + p)_x & = 0 \\
% 	      % 		      E_{t}        & + (v(E+p))_x & = 0
% 	      % 	      \end{aligned}
% 	      %       \end{cases}
% 	      %   \end{align}
% 	      \begin{align}
% 		      \begin{pmatrix} \rho\\\rho v \\ E \end{pmatrix}_{t}
% 		      +
% 		      \begin{pmatrix} \rho v\\\rho v^2 + p \\ v(E+p) \end{pmatrix}_{x}
% 		      =
% 		      \begin{pmatrix} 0\\ 0 \\ 0 \end{pmatrix}
% 	      \end{align}
% 	      $\rightarrow$ The pressure $p$ should be specified as a given function of 
% 	      mass density $\rho$, linear momentum $\rho v$ and/or energy $E$ in order to 
% 	      fulfill the \emph{closure} problem.
% 	      Such additional equation is called \emph{equation of state} (normally in fluid), or 
% 	      \emph{constitutive equation} (normally in solid). 
% 	\item \emph{Euler} equations of gas dynamics in 2D
% 	      \begin{align}
% 		      \begin{pmatrix} \rho\\ \rho u\\ \rho v \\ E \end{pmatrix}_{t}
% 		      +
% 		      \begin{pmatrix} \rho u\\ \rho u^2 + p \\ \rho uv \\ u(E+p) \end{pmatrix}_{x}
% 		      + 
% 		      \begin{pmatrix} \rho v\\ \rho uv \\ \rho v^2 + p \\ v(E+p) \end{pmatrix}_{y}
% 		      =
% 		      \begin{pmatrix} 0 \\ 0 \\ 0 \\ 0 \end{pmatrix}
% 	      \end{align}
% \end{enumerate}
% % \begin{recall}
% % 	Convexity
% % \end{recall}
% \begin{definition}
% 	Cauchy problem and Riemann problem
% \end{definition}
% Let $u$ be a conserved unknown quantity to be modelled and defined as follows
% \begin{align}\label{eq:conservedquantity}
% 	u:
% 	\begin{cases}
% 		\mathbb{R}\times\mathbb{R}^+ \rightarrow \mathbb{R}, \\
% 		(x,t) \mapsto u(x,t).
% 	\end{cases}
% \end{align}
% \emph{Cauchy's problem} is simply the pure initial value problem (IVP), e.g. 
% find a function $u$ holding \eqref{eq:conservedquantity} that is a solution of
% $\eqref{eq:cauchy}_1$ satisfying the initial condition $\eqref{eq:cauchy}_2$:
% \begin{equation}\label{eq:cauchy}
% 	\boxed{
% 		\begin{aligned}
% 			\partial_{t}u + \partial_{x}f(u) & = 0,        \\
% 			u(x,0)                           & = u_{0}(x).
% 		\end{aligned}
% 	}
% \end{equation}
% \emph{Riemann's problem} is simply the conservation law together with particular initial data
% consisting of two constant states separated by a single discontinuity, e.g.
% find a function $u$ holding \eqref{eq:conservedquantity} that is a solution of
% $\eqref{eq:riemann}_1$ satisfying the initial condition $\eqref{eq:riemann}_2$:
% \begin{equation}\label{eq:riemann}
% 	\boxed{
% 		\begin{aligned}
% 			\partial_{t}u + \partial_{x}f(u) & = 0,        \\
% 			u(x,0)                           & = u_{0}(x)=
% 			\begin{cases}
% 				u_{L}, \quad x<0, \\
% 				u_{R}, \quad x>0.
% 			\end{cases}
% 		\end{aligned}
% 	}
% \end{equation}
% % \begin{definition}
% % 	Hyperbolic
% % \end{definition}
% % Let $\Omega$ be 
% % \inputfig{floats/comparehyperbolic}{comparehyperbolic}
% %------------------------------------------------------------------------------
% \section{Characteristics}
% Let $f:\mathbb{R}\rightarrow\mathbb{R}$ be a $C^1$ function. We consider the \emph{Cauchy} problem
% \begin{align}\label{eq:charac}
% 	\frac{\partial u}{\partial t} + \frac{\partial}{\partial x}f(u) & = 0,        \qquad  x\in\mathbb{R}, t>0, \\
% 	u(x,0)                                                          & = u_{0}(x), \qquad  x\in\mathbb{R}.
% \end{align}
% By setting $a(u) = f'(u)$, and letting $u$ be a classical solution of \eqref{eq:charac},
% one obtains from \eqref{eq:charac} the \emph{non-conservative} form as follows
% \begin{equation}
% 	\frac{\partial u}{\partial t} + a(u)\frac{\partial u}{\partial x} = 0.
% \end{equation}
% The characteristic curves associated with \eqref{eq:charac} are defined as the integral curves
% of the differential equation
% \begin{equation}
% 	\frac{dx}{dt} = a(u(x(t),t)).
% \end{equation}
% \begin{proposition}
% 	Assume that u is a smooth solution of the Cauchy problem. The characteristic curves 
% 	are straight lines along which u is constant.
% \end{proposition}
% \begin{proof}
% 	Consider a characteristic curve passing through the point $(x_{0},0)$, i.e.
% 	a solution of the ordinary differential system
% 	\begin{align}
% 		\begin{cases}\displaystyle
% 			\frac{dx}{dt} & = a(u(x(t),t)), \\
% 			x(0)          & = x_{0}.
% 		\end{cases}
% 	\end{align}
% 	It exists at least on a small time interval $[0,t_{0})$. 
% 	Along such a curve, u is constant since
% 	\begin{align*}
% 		\frac{d}{dt}u(x(t),t) 
% 		 & = \frac{\partial u}{\partial t}(x(t),t)
% 		+ \frac{\partial u}{\partial x}(x(t),t) \frac{dx}{dt}(t)                                    \\
% 		 & =\left(\frac{\partial u}{\partial t} + a(u)\frac{\partial u}{\partial x}\right)(x(t),t) 
% 		= 0.
% 	\end{align*}
% 	Therefore, the characteristic curves are straight lines whose constant slopes depend
% 	on the initial data. As a result, the characteristic straight line passing through
% 	the point $(x_{0},0)$ is defined by the equation
% 	\begin{equation}
% 		\boxed{
% 			x = x_{0} + ta(u_{0}(x_{0}))
% 		}
% 	\end{equation}
% \end{proof}
% \begin{example}
% 	Determine characteristics of the inviscid Burgers' equation with initial conditions given as
% 	\begin{equation}
% 		u(x,0) = u_{0}(x) =
% 		\begin{cases}
% 			\begin{aligned}
% 				 & 1,   & x\leq 0,        &                        \\
% 				 & 1-x, & 0\leq x \leq 1, & \quad x\in \mathbb{R}, \\
% 				 & 0,   & x\geq 1.        & 
% 			\end{aligned}
% 		\end{cases}
% 	\end{equation}
% \end{example}
% By using the method of characteristics, the solution can be solved up to the time when
% those characteristics first intersect with each other, i.e. so-called breaking time or shock.
% Since the $f'(u) = u$ in the inviscid Burgers' equation, it yields the characteristics 
% passing through the point $(x_{0},t)$
% \begin{equation}
% 	x = x_{0} + t\,u_{0}(x_{0})
% \end{equation}
% which leads to
% \begin{equation}
% 	x(x_{0},t) = 
% 	\begin{cases}
% 		\begin{aligned}
% 			 & x_{0}+t,          & x_{0}\leq 0,          \\
% 			 & x_{0}+t(1-x_{0}), & 0\leq x_{0} \leq 1,   \\
% 			 & x_{0},            & x_{0}\geq 1.         
% 		\end{aligned}
% 	\end{cases}
% \end{equation}
% (Sketch)
% %------------------------------------------------------------------------------
% % \section{General strategy for solving the first-order PDE}
% % Supposed 
% %------------------------------------------------------------------------------
% % \section{Rankine-Hugoniot (RH) jump condition}

% % Characteristic curves
% % \begin{equation}
% % 	\dot{x} = f'(u)\\
% % \end{equation}
% % \begin{align*}
% % 	\frac{d}{dt}u(x(t),t) 
% % 	= \frac{\partial u}{\partial t}(x(t),t)
% % 	+ \frac{\partial u}{\partial x}(x(t),t) \frac{dx}{dt}(t)
% % \end{align*}


% % \inputfig{floats/multistructures}{multistructures}

\end{document}